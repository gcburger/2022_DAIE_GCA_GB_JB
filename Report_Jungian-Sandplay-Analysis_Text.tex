% Options for packages loaded elsewhere
\PassOptionsToPackage{unicode}{hyperref}
\PassOptionsToPackage{hyphens}{url}
%
\documentclass[
  10pt,
]{article}
\usepackage{amsmath,amssymb}
\usepackage{lmodern}
\usepackage{iftex}
\ifPDFTeX
  \usepackage[T1]{fontenc}
  \usepackage[utf8]{inputenc}
  \usepackage{textcomp} % provide euro and other symbols
\else % if luatex or xetex
  \usepackage{unicode-math}
  \defaultfontfeatures{Scale=MatchLowercase}
  \defaultfontfeatures[\rmfamily]{Ligatures=TeX,Scale=1}
\fi
% Use upquote if available, for straight quotes in verbatim environments
\IfFileExists{upquote.sty}{\usepackage{upquote}}{}
\IfFileExists{microtype.sty}{% use microtype if available
  \usepackage[]{microtype}
  \UseMicrotypeSet[protrusion]{basicmath} % disable protrusion for tt fonts
}{}
\makeatletter
\@ifundefined{KOMAClassName}{% if non-KOMA class
  \IfFileExists{parskip.sty}{%
    \usepackage{parskip}
  }{% else
    \setlength{\parindent}{0pt}
    \setlength{\parskip}{6pt plus 2pt minus 1pt}}
}{% if KOMA class
  \KOMAoptions{parskip=half}}
\makeatother
\usepackage{xcolor}
\usepackage[margin=1in]{geometry}
\usepackage{graphicx}
\makeatletter
\def\maxwidth{\ifdim\Gin@nat@width>\linewidth\linewidth\else\Gin@nat@width\fi}
\def\maxheight{\ifdim\Gin@nat@height>\textheight\textheight\else\Gin@nat@height\fi}
\makeatother
% Scale images if necessary, so that they will not overflow the page
% margins by default, and it is still possible to overwrite the defaults
% using explicit options in \includegraphics[width, height, ...]{}
\setkeys{Gin}{width=\maxwidth,height=\maxheight,keepaspectratio}
% Set default figure placement to htbp
\makeatletter
\def\fps@figure{htbp}
\makeatother
\setlength{\emergencystretch}{3em} % prevent overfull lines
\providecommand{\tightlist}{%
  \setlength{\itemsep}{0pt}\setlength{\parskip}{0pt}}
\setcounter{secnumdepth}{-\maxdimen} % remove section numbering
\ifLuaTeX
  \usepackage{selnolig}  % disable illegal ligatures
\fi
\IfFileExists{bookmark.sty}{\usepackage{bookmark}}{\usepackage{hyperref}}
\IfFileExists{xurl.sty}{\usepackage{xurl}}{} % add URL line breaks if available
\urlstyle{same} % disable monospaced font for URLs
\hypersetup{
  pdftitle={Descriptive \& Inferential Analysis of a Jungian Sandplay VR Project},
  pdfauthor={Gareth Burger \& James Bunt},
  hidelinks,
  pdfcreator={LaTeX via pandoc}}

\title{Descriptive \& Inferential Analysis of a Jungian Sandplay VR
Project}
\usepackage{etoolbox}
\makeatletter
\providecommand{\subtitle}[1]{% add subtitle to \maketitle
  \apptocmd{\@title}{\par {\large #1 \par}}{}{}
}
\makeatother
\subtitle{Normality testing of patient treatment - Report}
\author{Gareth Burger \& James Bunt}
\date{Tue, 27 December 22 at 16:33:14}

\begin{document}
\maketitle

{
\setcounter{tocdepth}{3}
\tableofcontents
}
\hypertarget{abstract}{%
\section{Abstract}\label{abstract}}

Based on the data provided of an experiment with 149 patients (1
removed) to test the effectiveness of VR treatment on PTSD, it is
reasonable to conclude that the VR treatment (both animated and static)
had a positive effect on PTSD, regardless of gender.

The experiment can be deemed as reliable since we have a randomised
controlled trial where participants within a pre-determined criteria
(young adults aged 18-25 years old) were sourced using random sampling
and randomly assigned to one of three test groups (1 control group and 2
different types of VR).

Observer reviewed test results were only considered since participant
self assessments are unreliable. Results comparing the pre and post PTSD
scores after VR treatment yielded a 15.01\% improvement compared to the
control group's improvement of 11.25\% (+3.8\%). Whilst all treatment
provided an overall improvement, a null hypothesis that VR treatment
does not impact PTSD can be rejected and an alternative hypothesis that
VR treatment will result in a greater reduction in PTSD compared to
traditional pyshchological therapy (control group) is suggested.

It is important to keep in mind that the above hypothesis is a
prediction and not a definitive statement about the relationship between
the VR treatment and reduced PTSD, but rather a starting point for
further investigation where more information is needed. Whilst the data
meets the assumptions required for a statistical test, it should also be
noted that the experiment information available is limited which limits
the reliability of findings. Specifically, the age of each participant
(i.e.~there maybe additional influence of age), the amount of time since
diagnosis (since both HSE and Mayo Clinic advise that some PTSD can
naturally reduce over time), whether the patient is receiving any
potentially conflicting treatment (i.e.~PTSD medication) and the time of
observation (it has been noted that observations are recorded at the
start and end of each day however this time of observation is not
included in the data provided) can impact the reliability of the
experiment which can not be validated at this time.

\hypertarget{introduction}{%
\section{Introduction}\label{introduction}}

According to the HSE and Mayo Clinic (2022) Post-traumatic stress
disorder (PTSD) is a mental health condition that is triggered by a
terrifying event --- either experiencing it or witnessing it. Symptoms
may include flashbacks, nightmares and severe anxiety, as well as
uncontrollable thoughts about the event (Mayo Clinic, 2022). PTSD
symptoms can be physical and/ or emotional and can vary over time or
vary from person to person (with or without treatment).

Treatment for PTSD can vary between psychological therapies (such as
cognitive behavioural therapy - CBT) and medication (such as
antidepressants). This study will only focus on the difference between
traditional pyscholigical therapy (CBT) and new VR versions (animated
and static) of a Jungian Sandplay (a therapeutic method that uses sand,
miniature objects, and image making within the context of the psychology
of Carl Jung (Maeve Dooley, October 2018).

There are several methods employed to assess patient PTSD including
observer-rated (OR), self- reported (CPSS), parental reported, and
physiological measurements however only OR pre and post scores are used
in this analysis due to reliability concerns of self recorded data
outside a controlled and standardised manner which is essential for a
robust experiment. Also note that specific focus will be applied on the
variance between the pre and post scores (impact).

\hypertarget{hypothesis}{%
\subsection{Hypothesis}\label{hypothesis}}

Null hypothesis is that VR treatments do not have a greater effect on
PTSD than the control group (tranditional CBT therapy) with an
alternative hypothesis that both animated and static VR treatments have
a positive effect on PTSD.

One possible hypothesis test that could be used to compare the
effectiveness of the different approaches used in the experiment is a
two-tailed independent samples t-test. This test could be used to
compare the mean change in PTSD levels (as measured by the pre-trial and
post-trial CPSS and OR scores) between the Static and Animated groups,
with the null hypothesis being that there is no significant difference
between the two groups. This test would help determine if the use of
animated VR content leads to significantly greater reductions in PTSD
levels than the use of static VR content.

The study consisted of 150 patients (evenly split between male and
female) divided into the groups listed below, using random sampling. A
population of 149 patients were analysed (1 removed due to missing data)
with 99 recieving treatment and 50 patients in the control group. All
patients were young adults in the age range of 18 - 25 years however
information on the exact age was not recorded but gender information was
recorded.

Note that there is no confirmation that participants are not receiving
any additional treatment which can easily impact the experiment results
therefore the hypothesis can not be validated

\hypertarget{design}{%
\subsection{Design}\label{design}}

There are a few important factors that should be considered when
conducting this experiment which has not been explicitly mentioned so
must be considered and noted as it impacts the reliability of analysis
and results.

Firstly, it is important to ensure that the study is conducted in a
controlled and standardized manner. This means that the sample of
patients should be selected using a random sampling method, and the
groups should be of equal size to avoid any potential bias. From the
information provided, since an appropriate random sampling approach has
been followed, this is deemed to be appropriate and reliable however it
should be noted that we have no information regarding other treatment
which the patients could be involved with (since all patients have
diagnosed PTSD, treatment is likely) which can directly impact the
reliability of these results.

Second, studies like this should be designed to control for potential
confounders, such as the patient's age, gender, and any other factors
that could affect the outcome of the experiment. Gender and age is
controlled within sampling which is sufficient however additional
information on potentially conflicting treatment (i.e.~mediciation) is
not mentioned. Within the data, whilst gender is recorded, information
on specific age and time of observation (i.e.~start of end of day) which
can impact results.

Third, it is important that this study should use valid and reliable
measures to assess the patients' PTSD levels, both at the start and end
of the study therefore all self recorded (CPSS) results have been
discounted since they are self reported and uncontrolled. Focusing
analysis on the observer rated results (OR) will help ensure that the
results of the study are accurate and unbiased so they can be compared
across the different test groups.

Fourth, it should be noted that it is important to ensure that the
therapists administering the treatment are trained and experienced in
using the VR app, as well as in providing traditional CBT which has not
been validated within the information provided. It is assumed that all
patients are receiving high-quality, consistent treatment so that any
differences in outcomes between the groups can be attributed to the
difference in treatment type.

Finally, the study should include a sufficient number of patients to
provide statistical power and to ensure that the results are
statistically significant. It has been deemed that the 150 patient
population (50 per treatment type) is sufficient for this analysis and
will help ensure that any observed differences between the groups can be
confidently attributed to the VR app rather than to chance.

\hypertarget{results}{%
\section{Results}\label{results}}

81\% of patients experienced improvement in their PTSD symptoms after VR
treatment and overall, all test groups experienced an improvement of
13.3\% median improvement (-0.72 points) which supports the hypothesis
that both animated and static VR therapy reduces PTSD. VR groups reduced
PTSD scores by 15.01\%, this equals to 3.8\% more than the control group
(-0.8 compared to -0.54) whihch means that we can also reject the null
hypothesis.

VR impact median reduced by -15.01\%. Pre score median was 6.16 with a
range of 2.06. Post score median was 5.43 with a range of 5.22.

No notable difference between VR groups was experienced (animated versus
static) with a 0.49\% variance in impact (between pre and post scores)

Control impact median reduced by -11.25\%. Pre score median was 5.98
with a range of 4.75. Post score median was 5.31 with a range of 5.44

Gender: no notable difference between gender for those groups recieving
VR Therapy (0.08 or 0.49\% range in impact) however note that females
were also receptive to the control experience (-12.5\% impact in control
compared to -15.0\% in VR)

Note: 70\% of the 10 participants who experienced the most impact, were
males from the static VR group

\emph{4. Provide descriptive statistics (graphs and tables) for any
assumptions made.}\\

\hypertarget{descriptive-statistics}{%
\subsection{Descriptive statistics}\label{descriptive-statistics}}

\emph{plots - box, bar, scatter, qq plots}

\hypertarget{inferential-statistics}{%
\subsection{Inferential statistics}\label{inferential-statistics}}

\emph{the inference that we are doing on our data requires that the data
is normally distributed}\\

\hypertarget{statistical-tests}{%
\subsection{Statistical tests}\label{statistical-tests}}

\hypertarget{magnitude-and-direction-of-results}{%
\subsection{Magnitude and direction of
results}\label{magnitude-and-direction-of-results}}

\emph{6. Determine the 95\% confidence interval for the population mean
of each group, and the 95\% confidence interval for the difference
between the means of any two groups for a variable of your choice.}\\

We are 95\% confident that {[}PE + x, PE - x{]} of all animated group

\emph{7. Determine the degree of correlation between any explanatory and
response variable of your choice.}\\

\emph{do student t test - see diamond quality video}

\hypertarget{discussion}{%
\section{Discussion}\label{discussion}}

\emph{5. Analyse the data to provide the hypothesis testing
conclusion.}\\

Outline findings and relation to the hypothesis

Limitations (if confounding variables are clearly identified by your
group)

\begin{enumerate}
\def\labelenumi{\arabic{enumi}.}
\tightlist
\item
  Determine whether the data provided are appropriate for the test(s)
  available and whether any data cleaning is required.
\end{enumerate}

\begin{itemize}
\tightlist
\item
  Cleaning of the data was required, specifically; 1 row had to be
  removed due to a missing test group, 1 row had a mispelt gender which
  was corrected and 1 row had a CPSS post result of 12 (2 over the
  maximum scale) however this row was kept in the set as only the OR
  data was considered in this analysis.
\item
  Is the data appropriate for the tests available? (is 50 big enough for
  each test group?)
\end{itemize}

\begin{enumerate}
\def\labelenumi{\arabic{enumi}.}
\setcounter{enumi}{1}
\item
  Formulate a single hypothesis test to be used to compare the
  effectiveness of the approaches used during the experiment.
\item
  Determine if the data meet the assumptions required by any statistical
  test.
\item
  Provide descriptive statistics (graphs and tables) for any assumptions
  made.
\item
  Analyse the data to provide the hypothesis testing conclusion.
\item
  Determine the 95\% confidence interval for the population mean of each
  group, and the 95\% confidence interval for the difference between the
  means of any two groups for a variable of your choice.
\item
  Determine the degree of correlation between any explanatory and
  response variable of your choice.
\end{enumerate}

\hypertarget{references}{%
\section{References}\label{references}}

\begin{itemize}
\tightlist
\item
  Dooley, M. (2018) `Jungian Sandplay'. Available at:
  \url{https://maevedooley.ie/sandplay/}. Accessed on 20th December
  2022.
\item
  HSE (2022). `Post-traumatic stress disorder (PTSD)'. Available at:
  \url{https://www2.hse.ie/conditions/ptsd/treatment/}. Accessed on 21
  December 2022.
\item
  Mayo Clinic (2022). `Post-traumatic stress disorder (PTSD)'. Available
  at:
  \url{https://www.mayoclinic.org/diseases-conditions/post-traumatic-stress-disorder/symptoms-causes/syc-20355967}.
  Accessed on 21 December 2022.
\item
\end{itemize}

\end{document}
